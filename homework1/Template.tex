\documentclass[a4paper, justified]{tufte-handout}

\input{hw-preamble}
%%%%%%%%%%%%%%%%%%%%
\title{Homework}
\me{赵超懿}{191870271}{}{}
\date{\zhtoday}
%%%%%%%%%%%%%%%%%%%%
\begin{document}
\maketitle


%%%%%%%%%%%%%%%
\begin{problem}[P17 1]
    证明$|PROP| = \aleph_0 $
\end{problem}

\begin{solution}
    记最短构造序列长度为n的命题构成的集合为$T_n$,对n进行归纳:\\
    $T_1=|PS| = \aleph_0$,假设当n=k时结论成立,当n=k+1时,$|T_{k+1}|=|T_{k}|+3*|T_{k}\times T_{k}|| = \aleph_0$,所以对任意命题P都存在于某一个$T_n$中,$T_1,T_2,T_3,T_4,\dots,T_n,\dots$为可数个无穷可数集合,故$|PROP|=|\bigcup\limits_{i=1}^{\infty}T_i|=\aleph_0$
\end{solution}

\begin{problem}[P17 3-(a)(b)]
    证明以下命题永真\\
    (a) $A\rightarrow A$\\
    (b) $((A\rightarrow B)\wedge (B\rightarrow C)) \rightarrow (A\rightarrow C)$
\end{problem}

\begin{solution}
    (a)
    \begin{table}[http]
        \begin{tabular}{|l|l|}
            \hline
        A & $A\rightarrow A$ \\
        \hline
        T & T                \\
        \hline
        F & T               \\
        \hline
        \end{tabular}
    \end{table}\\

    (b)\\
    \begin{table}[http]
        \begin{tabular}{|c|c|c|c|}
        \hline
        A & B & C & $((A\rightarrow B)\wedge (B\rightarrow C)\rightarrow (A\rightarrow C)$ \\ \hline
        T & T & T & T                                                                      \\ \hline
        T & T & F & T                                                                      \\ \hline
        T & F & T & T                                                                      \\ \hline
        T & F & F & T                                                                      \\ \hline
        F & T & T & T                                                                      \\ \hline
        F & T & F & T                                                                      \\ \hline
        F & F & T & T                                                                      \\ \hline
        F & F & F & T                                                                      \\ \hline
        \end{tabular}
        \end{table}


\end{solution}


\begin{problem}[P17 6-(a)(b)]
    设习题3中的命题为$A'$,在$G'$中证明$\vdash A'$\\
    (a) $A\rightarrow A$\\
    (b) $((A\rightarrow B)\wedge (B\rightarrow C)) \rightarrow (A\rightarrow C)$
\end{problem}

\begin{solution}
(a)\\
\[
    \underline{A\vdash A}_{(\rightarrow R)}
\]
\[
    \vdash A\rightarrow A
\]
(b)\\
\[
    \underline{A\vdash A,B,C}_{(\rightarrow R)}\qquad \underline{A,C\vdash A,C}_{(\rightarrow R)}\qquad \qquad \qquad \qquad \qquad \qquad \qquad \underline{A,B,C\vdash C}_{(\rightarrow R)}
\]
\[
    \underline{\vdash A,A\rightarrow C,B\qquad C\vdash A,A\rightarrow C}_{(\rightarrow L)}\qquad \qquad \underline{B\vdash B,A\rightarrow C\qquad B,C\vdash A\rightarrow C}_{(\rightarrow L)}
\]
\[
    \underline{B\rightarrow C\vdash A,A\rightarrow C\qquad \qquad \qquad\qquad \qquad B\rightarrow C,B\vdash A\rightarrow C}_{(\rightarrow L)}
\]
\[
    \underline{((A\rightarrow B)\wedge (B\rightarrow C)) \vdash (A\rightarrow C)}_{(\rightarrow R)}
\]
\[
    \vdash((A\rightarrow B)\wedge (B\rightarrow C)) \rightarrow (A\rightarrow C)
\]
\end{solution}

\begin{problem}[P17 7]
    证明在$G'$中$\vdash(P\rightarrow Q)\vee R$不可证,这里$P,Q,R\in PS$
\end{problem}

\begin{solution}
\[
    \underline{P\vdash Q,R}_{(\rightarrow R)}
\]
\[
    \underline{\vdash(P\rightarrow Q), R}_{(\vee R)}
\]
\[
    \vdash(P\rightarrow Q)\vee R
\]
\end{solution}

\end{document}